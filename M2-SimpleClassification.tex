%%%%%%%%%%%%%%%%%%%%%%%%%%%%%%%%%%%%%%%%%%%%%%%%%%%%%%%%%%%%
%%  This Beamer template was created by Cameron Bracken.

\documentclass[xcolor=x11names,compress, aspectratio=169]{beamer}
%\documentclass[xcolor=x11names,compress, handhouts, aspectratio=169]{beamer}
%% General document
\usepackage{graphicx, subfig}
%% Beamer Layout
\useoutertheme[subsection=false,shadow]{miniframes}
\useinnertheme{default}
\usefonttheme{serif}
\usepackage{palatino}

%%%%%%% Mes Packages %%%%%%%%%%%%%%%%
%\usepackage[french]{babel}
\usepackage[T1]{fontenc}
\usepackage{color}
\usepackage{xcolor}
\usepackage{dsfont} % Pour indicatrice
\usepackage{url}
\usepackage{multirow}
%remove the icon
\setbeamertemplate{bibliography item}{}

%remove line breaks
\setbeamertemplate{bibliography entry title}{}
\setbeamertemplate{bibliography entry location}{}
\setbeamertemplate{bibliography entry note}{}

%% ------ MEs couleurs --------
\definecolor{vert}{rgb}{0.1,0.7,0.2}
\definecolor{brique}{rgb}{0.7,0.16,0.16}
\definecolor{gris}{rgb}{0.7, 0.75, 0.71}
\definecolor{twitterblue}{rgb}{0, 0.42, 0.58}
\definecolor{airforceblue}{rgb}{0.36, 0.54, 0.66}
\definecolor{siap}{RGB}{3,133, 168}


%%%%%%%%%%%%%%%%% BEAMER PACKAGE %%%%%%%

\setbeamerfont{title like}{shape=\scshape}
\setbeamerfont{frametitle}{shape=\scshape}

\setbeamercolor*{lower separation line head}{bg=DeepSkyBlue4}
\setbeamercolor*{normal text}{fg=black,bg=white}
\setbeamercolor*{alerted text}{fg=red}
\setbeamercolor*{example text}{fg=black}
\setbeamercolor*{structure}{fg=black}
\setbeamercolor*{palette tertiary}{fg=black,bg=black!10}
\setbeamercolor*{palette quaternary}{fg=black,bg=black!10}

\renewcommand{\(}{\begin{columns}}
\renewcommand{\)}{\end{columns}}
\newcommand{\<}[1]{\begin{column}{#1}}
\renewcommand{\>}{\end{column}}

% Path for the graphs
\graphicspath{{Graphics/}{../../../../Visualisation/Presentations/Graphics/}{../../Visualisation/Presentations/Graphics/}{c:/Gitmain/MLCourse/UNML/Module2/M2-SimpleClassification_files/figure-html/}  }

%remove navigation symbols
\setbeamertemplate{navigation symbols}{}


% Natbib for clean bibliography
\usepackage[comma,authoryear]{natbib}

\begin{document}
%%% Title page %%%%%
\begin{frame}
\Large{ \color{siap}{Machine Learning for Official Statistics and SDGs}}

\hspace{1cm}

\color{brique}{\huge{Supervised Classification}}

\hspace{2cm}
\begin{center}

\includegraphics[height=0.10\textwidth]{SIAP_logo_Big.png}

\end{center}
\end{frame}

%%%%%%%%%%%%%%%%%%%%%%%%%%%%%%%

%%%  ----------- Dataviz  Definition  -----------%%%

\section{Introduction}

\begin{frame} % Cover slide

\frametitle{\textcolor{brique}{[ Classification ]}}
What is a classification problem? 
\pause
\begin{itemize}[<+->]
  \item The goal of classification is to understand why an observation belongs to a certain category
  \item The interest is on $y$ that takes discrete values:  0/1, high school/prilmary school/no education; urban/rural
  \item There may be variables explaining why we observe that $y$ belongs to a particular category
  \item A classifier is a tool that  provides a classification for $y$
\end{itemize}
\end{frame}



\begin{frame} % Cover slide

\frametitle{\textcolor{brique}{[ Supervised \textit{vs} unsupervised Classification ]}}

\pause
\begin{itemize}[<+->]
  \item In \textbf{supervised }classification, we observe the class for  $y$
  \item[] One may learn, for that information and estimate the impact of other variables on that classification (regression)
   \item In \textbf{unsupervised }classification, we observe a set of variables for each observation
  \item[]  The goal is to classify observations from those variables (clustering) without having any information of what a category means. 
  \item We'll focus on \textbf{supervised }classification
\end{itemize}
\end{frame}



\begin{frame} % Cover slide
\frametitle{\textcolor{brique}{[ Classification:  An example ]}}
\pause
\begin{itemize}[<+->]
  \item You observe households that are either in \textit{Urban} or \textit{Rural} areas (represented here by the colors) and one variable (feature): \textit{Education}. 
  \item[] \begin{center}\includegraphics[width = 0.5\textwidth]{univariatefig-1.png} \end{center}
  \item A classifier here would determine the value of Education that separate "Rural" from  "Urban" 
  \item A classifier is "if $x \geq x_t$  then $y$ is categorized as Urban" 
\end{itemize}
\end{frame}



\begin{frame} % Cover slide
\frametitle{\textcolor{brique}{[ Classification:  A 2-D example ]}}
\pause
\begin{itemize}[<+->]
  \item You observe households that are either in \textit{Urban} or \textit{Rural} areas (represented here by the colors) and \textbf{two} variables (features): \textit{Education} and \textit{Income}
  \item[] \begin{center}\includegraphics[width = 0.5\textwidth]{bivariatefig-1.png} \end{center}
\end{itemize}
\end{frame}




\begin{frame} % Cover slide
\frametitle{\textcolor{brique}{[ Classification:  A 2-D example ]}}
\pause
\begin{itemize}[<+->]
  \item[] \begin{center}\includegraphics[width = 0.3\textwidth]{bivariatefig-1.png} \end{center}
   \item A classifier here would determine a rule with both \textit{Education}  and \textit{Income} to separate "rural" from "Urban"
  \item The rule can be based on a linear relationship between \textit{Education}  and \textit{Income} or can be non linear.
\end{itemize}
\end{frame}


\begin{frame} % Cover slide
\frametitle{\textcolor{brique}{[ Classification:  A 2-D example ]}}
\pause
\begin{itemize}[<+->]
  \item The logit is an example of linear classifier
  \item[] \begin{center}\includegraphics[width = 0.5\textwidth]{Logit_Frontier-1.png} \end{center}
  \item The rule that separated the two classes is $ x'\beta \geq t_0$ with $t_0$ a known threshold
  \item[] or $\beta_1 Education + \beta_2 Income \geq t_0$ 
 
\end{itemize}
\end{frame}

\begin{frame} % Cover slide
\frametitle{\textcolor{brique}{[ Classification:  A 2-D example ]}}
\pause
\begin{itemize}[<+->]
  \item The quadratic logit is an example of non-linear classifier
  \item[] \begin{center}\includegraphics[width = 0.5\textwidth]{QLogit_Frontier-1.png} \end{center}
  \item The rule that separated the two classes is non linear in the variables textit{Education}  and \textit{Income} 
\end{itemize}
\end{frame}


\begin{frame} % Cover slide
\frametitle{\textcolor{brique}{[ Classification:  A 2-D example ]}}
\pause
\begin{itemize}[<+->]
  \item Other examples can be very non linear
  \item[] \begin{center}\includegraphics[width = 0.5\textwidth]{PlotBayesK-1.png} \end{center}
  \item It is hard to understand how the two classes  are built using \textit{Education}  and \textit{Income}
\end{itemize}
\end{frame}


\begin{frame} % Cover slide
\frametitle{\textcolor{brique}{[ Classification:  A 2-D example ]}}
\pause
\begin{itemize}[<+->]
  \item Other examples can be very non linear
  \item[] \begin{center}\includegraphics[width = 0.5\textwidth]{PlotQDA-1.png} \end{center}
  \item It is hard to understand how the two classes  are built using \textit{Education}  and \textit{Income}
\end{itemize}
\end{frame}


\begin{frame} % Cover slide
\frametitle{What did we learn?}
 \begin{itemize}
  \item<+-> Is the line fitting the data?
   \begin{itemize}
      \item<+->[] One may always find a relation between two variables
      \item<+->[] Statistical test help asses the significance of a variable
  \end{itemize}
  \item<+-> How good is the linear adjustment: $R^2$
  \begin{itemize}
      \item<+->[] $$R^2 = \frac{TSS- RSS}{TSS}$$
      \item<+->[] with:  $TSS= \sum_{i=1}^n (y_i -\bar{y})^2 $ and $RSS= \sum_{i=1}^n (y_i - \widehat{f}(x_i))^2 $
 \end{itemize}
 \item<+-> $R^2$ is a very popular measure. The closer to 1, the better.
 \item<+->  $R^2$  can be very misleading
 \end{itemize}
\end{frame}



%%%%%%%%%%%%%%%%%%%%%%%%%%%%%%%


\end{document}

\begin{frame} % Cover slide
\frametitle{ }
\pause
 \begin{itemize}[<+->]
  \item[]
  \item
\end{itemize}
\end{frame}

%%%%%%%%%%%%%%% Last Slide %%%%%%%%%%%%%%%%

\begin{frame}[allowframebreaks]%in case more than 1 slide needed
\frametitle{References}
    {\footnotesize
    %\bibliographystyle{authordate1}
    \bibliographystyle{apalike}
    \bibliography{../../../Visualisation/Visu}
    }
\end{frame}
\end{document}

%\bibliographystyle{authordate1}
%\bibliography{c:/Chris/Visualisation/Visu}
%\end{frame}
