%%%%%%%%%%%%%%%%%%%%%%%%%%%%%%%%%%%%%%%%%%%%%%%%%%%%%%%%%%%%
%%  This Beamer template was created by Cameron Bracken.

\documentclass[xcolor=x11names,compress, handhouts]{beamer}
%% General document
\usepackage{graphicx, subfig}
%% Beamer Layout
\useoutertheme[subsection=false,shadow]{miniframes}
\useinnertheme{default}
\usefonttheme{serif}
\usepackage{palatino}

%%%%%%% Mes Packages %%%%%%%%%%%%%%%%
%\usepackage[french]{babel}
\usepackage[T1]{fontenc}
\usepackage{color}
\usepackage{xcolor}
\usepackage{dsfont} % Pour indicatrice
\usepackage{url}
\usepackage{multirow}

%remove the icons
\setbeamertemplate{bibliography item}{}

%remove line breaks
\setbeamertemplate{bibliography entry title}{}
\setbeamertemplate{bibliography entry location}{}
\setbeamertemplate{bibliography entry note}{}

%% ------ MEs couleurs --------
\definecolor{vert}{rgb}{0.1,0.7,0.2}
\definecolor{brique}{rgb}{0.7,0.16,0.16}
\definecolor{gris}{rgb}{0.7, 0.75, 0.71}
\definecolor{twitterblue}{rgb}{0, 0.42, 0.58}
\definecolor{airforceblue}{rgb}{0.36, 0.54, 0.66}
\definecolor{siap}{RGB}{3,133, 168}


%%%%%%%%%%%%%%%%% BEAMER PACKAGE %%%%%%%

\setbeamerfont{title like}{shape=\scshape}
\setbeamerfont{frametitle}{shape=\scshape}

\setbeamercolor*{lower separation line head}{bg=DeepSkyBlue4}
\setbeamercolor*{normal text}{fg=black,bg=white}
\setbeamercolor*{alerted text}{fg=red}
\setbeamercolor*{example text}{fg=black}
\setbeamercolor*{structure}{fg=black}
\setbeamercolor*{palette tertiary}{fg=black,bg=black!10}
\setbeamercolor*{palette quaternary}{fg=black,bg=black!10}

\renewcommand{\(}{\begin{columns}}
\renewcommand{\)}{\end{columns}}
\newcommand{\<}[1]{\begin{column}{#1}}
\renewcommand{\>}{\end{column}}

% Path for the graphs
\graphicspath{{Graphics/}{../../../../Visualisation/Presentations/Graphics/}{../../Visualisation/Presentations/Graphics/}
{c:/Gitmain/MLCourse/UNML/Module4/M4-0-SimpleTrees_files/figure-html/}{c:/Gitmain/MLCourse/UNML/Module2/M2-1-SimpleClassification_files/figure-html/}{c:/GitMain/MLCourse/UNML/Module0/M0_files/figure-html/}
}

%remove navigation symbols
\setbeamertemplate{navigation symbols}{}


% Natbib for clean bibliography
\usepackage[comma,authoryear]{natbib}

\begin{document}
%%% Title page %%%%%
\begin{frame}
\Large{ \color{siap}{Machine Learning for Official Statistics \& SDGs}}

\hspace{1cm}

\color{brique}{\huge{Decision Trees}}

\hspace{2cm}
\begin{center}

\includegraphics[height=0.10\textwidth]{SIAP_logo_Big.png}

\end{center}
\end{frame}

%%%%%%%%%%%%%%%%%%%%%%%%%%%%%%%

%% http://www.sthda.com/english/articles/37-model-selection-essentials-in-r/153-penalized-regression-essentials-ridge-lasso-elastic-net/
%%

%%%  ----------- Dataviz  Definition  -----------%%%
\section{Introduction}

\begin{frame} % Cover slide

\frametitle{\textcolor{brique}{[ Decision Trees ]}}
\textit{Decision Trees} are methods used for classification or regression analysis. $\hookrightarrow$ Focus on classification\\
\begin{center} \includegraphics[width = 0.5\textwidth]{FullTree.png} \end{center}
\pause
\begin{itemize}[<+->]
  \item Trees split the space into non-overlapping spaces
  \item Used to assign/predict a \textit{class} following conditions 
  \item Optimally, final classification is homogenous
\end{itemize}
\end{frame}

\section{What's in a tree?}

\begin{frame} % Cover slide
\frametitle{\textcolor{brique}{[ What's in a tree? ]}}
Trees have a very simple structure and are easy to understand: 
\begin{center} \includegraphics[width = 0.6\textwidth]{FullTree.png} \end{center}
\pause
\begin{itemize}[<+->]
  \item[] Trees have:
  \begin{itemize}[<+->]
    \item \textbf{Nodes} where splitting decisions are done
    \item \textbf{Branches}  following conditions
    \item \textbf{Leaves} are terminal nodes of the classification
  \end{itemize}
\end{itemize}
\end{frame}
 \section{How to build a tree?}

\begin{frame} 
\frametitle{\textcolor{brique}{[ How to construct a tree? ]}}

\begin{center} \includegraphics[width = 0.6\textwidth]{FullTree.png} \end{center}
\pause
\begin{itemize}[<+->]
    \item Trees are based on recursive binary splits
    \item Each node is a condition on a variable, based on a threshold. 
    \item Each node separates the observations in two sets 
\end{itemize}
\end{frame}



\begin{frame}
\frametitle{\textcolor{brique}{[ How to construct a tree? ]}}
 \hfill \includegraphics[width = 0.2\textwidth]{FullTree.png} \\
 
We need several tools to build a tree
\pause
\begin{itemize}[<+->]
    \item  A method to choose the decision variable (one per node)
    \item  A criterion to define best threshold  
    \item  A criterion to measure the quality of each split 
    \item  A way to decide when to stop (the terminal node becomes a leaf)
\end{itemize}
\end{frame}

\begin{frame}
\frametitle{\textcolor{brique}{[ Quality of a tree ]}}
At each node, one can measure the \textit{purity} of each split
\pause
\begin{itemize}[<+->]
    \item  Misclassification error rate
    \item  The Gini coefficient measures the purity in each node
$$
  D_{\kappa} = \sum_{m=1}^M  \widehat{p}_{m \kappa} (1-\widehat{p}_{m
  \kappa})
$$
  where $\widehat{p}_{m \kappa}$ is the proportion of class $m$ in
  node $\kappa$.  
    \item  The entropy or information:
 $$
  D_{\kappa} =   - \sum_{m=1}^M  \widehat{p}_{m \kappa}  \log
  \widehat{p}_{m \kappa}
  $$
  \item[Note:] If in the each node there is only one class   represented, then $D_{\kappa}$ is zero.  
  
\end{itemize}
\end{frame}


\section{Step-by-Step}

\begin{frame}
\frametitle{\textcolor{brique}{[ Example on a simple tree ]}}
Let us see how this tree is constructed:\\
\includegraphics[width = 0.9\textwidth]{rparttree-1.png} \\
4 nodes leading to final leaves $\hookrightarrow$  Depth = 5 
\end{frame}

\begin{frame}
\frametitle{\textcolor{brique}{[ Example on a simple tree ]}}
The problem is a 2D space: Education and Income and a classification in Urban vs Rural\\
\includegraphics[width = 0.9\textwidth]{Step0-1.png}
\end{frame}

\begin{frame} % Cover slide
\frametitle{\textcolor{brique}{[ Example on a simple tree ]}}
\hfill \includegraphics[width = 0.3\textwidth]{rparttree-1.png} 
\begin{itemize}
\item[]
  \only<2> {\includegraphics[width = 0.8\textwidth]{Step1-1.png} \\ }
  \only<2> {First the boundary decision line}
  \only<3> { \includegraphics[width = 0.8\textwidth]{Step1-2.png} \\ }
  \only<3> {The space below the line is classified as  rural}
  \only<4> {\includegraphics[width = 0.8\textwidth]{Step2-1.png} \\ }
  \only<4> {Second boundary decision line}
  \only<5> {\includegraphics[width = 0.8\textwidth]{Step2-2.png} \\ }
  \only<5> {The space on the left of the line is classified as  rural}
  \only<6> {\includegraphics[width = 0.8\textwidth]{Step3-1.png} \\ }
  \only<6> {Third boundary decision line}
  \only<7> {\includegraphics[width = 0.8\textwidth]{Step3-2.png} \\ }
  \only<7> {The space on the right of the line is classified as  Urban}
  \only<8> {\includegraphics[width = 0.8\textwidth]{Step4-1.png} \\ }
  \only<8> {Fourth boundary decision line}
  \only<9> {\includegraphics[width = 0.8\textwidth]{Step4-2.png} \\ }
  \only<9> {The space above the line is classified as  Urban}
  \only<10> {\includegraphics[width = 0.8\textwidth]{Step4-3.png} \\ }
  \only<10> {Finally, the remaining space is classified as rural}
\end{itemize}
\end{frame}




\end{document}
%%%%%%%%%%%%%%% Last Slide %%%%%%%%%%%%%%%%

\begin{frame}[allowframebreaks]%in case more than 1 slide needed
\frametitle{References}
    {\footnotesize
    %\bibliographystyle{authordate1}
    \bibliographystyle{apalike}
    \bibliography{../../../Visualisation/Visu}
    }
\end{frame}
\end{document}

%\bibliographystyle{authordate1}
%\bibliography{c:/Chris/Visualisation/Visu}
%\end{frame}
