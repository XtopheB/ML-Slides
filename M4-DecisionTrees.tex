%%%%%%%%%%%%%%%%%%%%%%%%%%%%%%%%%%%%%%%%%%%%%%%%%%%%%%%%%%%%
%%  This Beamer template was created by Cameron Bracken.

\documentclass[xcolor=x11names,compress, handhouts]{beamer}
%% General document
\usepackage{graphicx, subfig}
%% Beamer Layout
\useoutertheme[subsection=false,shadow]{miniframes}
\useinnertheme{default}
\usefonttheme{serif}
\usepackage{palatino}

%%%%%%% Mes Packages %%%%%%%%%%%%%%%%
%\usepackage[french]{babel}
\usepackage[T1]{fontenc}
\usepackage{color}
\usepackage{xcolor}
\usepackage{dsfont} % Pour indicatrice
\usepackage{url}
\usepackage{multirow}

%remove the icons
\setbeamertemplate{bibliography item}{}

%remove line breaks
\setbeamertemplate{bibliography entry title}{}
\setbeamertemplate{bibliography entry location}{}
\setbeamertemplate{bibliography entry note}{}

%% ------ MEs couleurs --------
\definecolor{vert}{rgb}{0.1,0.7,0.2}
\definecolor{brique}{rgb}{0.7,0.16,0.16}
\definecolor{gris}{rgb}{0.7, 0.75, 0.71}
\definecolor{twitterblue}{rgb}{0, 0.42, 0.58}
\definecolor{airforceblue}{rgb}{0.36, 0.54, 0.66}
\definecolor{siap}{RGB}{3,133, 168}


%%%%%%%%%%%%%%%%% BEAMER PACKAGE %%%%%%%

\setbeamerfont{title like}{shape=\scshape}
\setbeamerfont{frametitle}{shape=\scshape}

\setbeamercolor*{lower separation line head}{bg=DeepSkyBlue4}
\setbeamercolor*{normal text}{fg=black,bg=white}
\setbeamercolor*{alerted text}{fg=red}
\setbeamercolor*{example text}{fg=black}
\setbeamercolor*{structure}{fg=black}
\setbeamercolor*{palette tertiary}{fg=black,bg=black!10}
\setbeamercolor*{palette quaternary}{fg=black,bg=black!10}

\renewcommand{\(}{\begin{columns}}
\renewcommand{\)}{\end{columns}}
\newcommand{\<}[1]{\begin{column}{#1}}
\renewcommand{\>}{\end{column}}

% Path for the graphs
\graphicspath{{Graphics/}{../../../../Visualisation/Presentations/Graphics/}{../../Visualisation/Presentations/Graphics/}
{c:/Gitmain/MLCourse/UNML/Module4/M4-0-SimpleTrees_files/figure-html/}{c:/Gitmain/MLCourse/UNML/Module4/M4-1-DecisionTrees_files/figure-html/}  {c:/Gitmain/MLCourse/UNML/Module2/M2-1-SimpleClassification_files/figure-html/}{c:/GitMain/MLCourse/UNML/Module0/M0_files/figure-html/}
}

%remove navigation symbols
\setbeamertemplate{navigation symbols}{}


% Natbib for clean bibliography
\usepackage[comma,authoryear]{natbib}

\begin{document}
%%% Title page %%%%%
\begin{frame}
\Large{ \color{siap}{Machine Learning for Official Statistics \& SDGs}}

\hspace{1cm}

\color{brique}{\huge{Decision Trees}}

\hspace{2cm}
\begin{center}

\includegraphics[height=0.10\textwidth]{SIAP_logo_Big.png}

\end{center}
\end{frame}

%%%%%%%%%%%%%%%%%%%%%%%%%%%%%%%

%% http://www.sthda.com/english/articles/37-model-selection-essentials-in-r/153-penalized-regression-essentials-ridge-lasso-elastic-net/
%%

%%%  ----------- Dataviz  Definition  -----------%%%
\section{Introduction}

\begin{frame} % Cover slide

\frametitle{\textcolor{brique}{[ Decision Trees ]}}
\textit{Trees} are method for classification or regression analysis.\\
 $\hookrightarrow$ Focus on classification\\
\begin{center} \includegraphics[width = 0.5\textwidth]{rparttree-1.png} \end{center}
\pause
\begin{itemize}
  \item Trees split the space into non-overlapping spaces
  \item Used to assign/predict a \textit{class} following conditions
  \item Optimally, final classification is homogenous
\end{itemize}
\end{frame}

\section{What's in a tree?}

\begin{frame} % Cover slide
\frametitle{\textcolor{brique}{[ What's in a tree? ]}}
Trees have a very simple structure and are easy to understand:
\begin{center} \includegraphics[width = 0.6\textwidth]{rparttree-1.png} \end{center}
\pause
\begin{itemize}
  \item[] Trees have:
  \begin{itemize}[<+->]
    \item \textbf{Nodes} where splitting decisions are done
    \item \textbf{Branches}  following conditions
    \item \textbf{Leaves} are terminal nodes of the classification
  \end{itemize}
\end{itemize}
\end{frame}


\begin{frame}
\frametitle{\textcolor{brique}{[ What's in a tree? ]}}
How to build a tree?
\begin{center} \includegraphics[width = 0.6\textwidth]{rparttree-1.png} \end{center}
\pause
\begin{itemize}
    \item Trees are based on recursive binary splits
    \item Each node uses a threshold on a variable
    \item Each node separates the observations in two sets
\end{itemize}
\end{frame}


\section{Step-by-Step}

\begin{frame}
\frametitle{\textcolor{brique}{[ Example on a simple tree ]}}
Let us see how this tree is constructed:\\
\includegraphics[width = 0.9\textwidth]{rparttree-1.png} \\
4 nodes leading to final leaves $\hookrightarrow$  Depth = 5
\end{frame}

\begin{frame}
\frametitle{\textcolor{brique}{[ Example on a simple tree ]}}
The problem is a 2D space\\
\includegraphics[width = 0.9\textwidth]{Step0-1.png}
\end{frame}

\begin{frame} % Cover slide
\frametitle{\textcolor{brique}{[ Example on a simple tree ]}}
\hfill \includegraphics[width = 0.3\textwidth]{rparttree-1.png}
\begin{itemize}
\item[]
   \only<2> {\includegraphics[width = 0.8\textwidth]{Step0-1.png} \\ }
   \only<2> {How to split the (Education,Income) space?}
   \only<3> {\includegraphics[width = 0.8\textwidth]{Step1-1.png} \\ }
   \only<3> {The first boundary decision line}
   \only<4> { \includegraphics[width = 0.8\textwidth]{Step1-2.png} \\ }
   \only<4> {The space below the line is classified as  rural}
   \only<5> {\includegraphics[width = 0.8\textwidth]{Step2-1.png} \\ }
   \only<5> {Second boundary decision line}
   \only<6> {\includegraphics[width = 0.8\textwidth]{Step2-2.png} \\ }
   \only<6> {The space on the left of the line is classified as  rural}
   \only<7> {\includegraphics[width = 0.8\textwidth]{Step3-1.png} \\ }
   \only<7> {Third boundary decision line}
   \only<8> {\includegraphics[width = 0.8\textwidth]{Step3-2.png} \\ }
   \only<8> {The space on the right of the line is classified as  Urban}
   \only<9> {\includegraphics[width = 0.8\textwidth]{Step4-1.png} \\ }
   \only<9> {Fourth boundary decision line}
   \only<10>{\includegraphics[width = 0.8\textwidth]{Step4-2.png} \\ }
   \only<10>{The space above the line is classified as  Urban}
   \only<11>{\includegraphics[width = 0.8\textwidth]{Step4-3.png} \\ }
   \only<11>{Finally, the remaining space is classified as rural}
\end{itemize}
\end{frame}

\section{How to build a tree?}

\begin{frame}
\frametitle{\textcolor{brique}{[ How to build a tree? ]}}
 \hfill \includegraphics[width = 0.2\textwidth]{rparttree-1.png} \\

We need several tools to build a tree:
\pause
\begin{itemize}[<+->]
    \item  A method to choose the decision variable (one per node)
    \item  A criterion to define best threshold
    \item  A criterion to measure the quality of each split
    \item  A way to decide when to stop (the terminal node becomes a leaf)
\end{itemize}
\end{frame}

\begin{frame}
\frametitle{\textcolor{brique}{[ Tools to build a tree ]}}
At each node, one can measure the \textit{purity} of each split
\pause
\begin{itemize}[<+->]
    \item  Misclassification error rate
    \item  The Gini coefficient measures the purity in each node $\kappa$
$$
  D_{\kappa} = \sum_{m=1}^M  \widehat{p}_{m \kappa} (1-\widehat{p}_{m
  \kappa})
$$
  where $\widehat{p}_{m \kappa}$ is the proportion of class $m$ in
  node $\kappa$.
    \item  The entropy or information:
 $$
  D_{\kappa} =   - \sum_{m=1}^M  \widehat{p}_{m \kappa}  \log
  \widehat{p}_{m \kappa}
  $$
  \item We expect that there is an Information gain from the splitting
  $$Information\;Gain = Entropy_{\; Before} - Entropy_{\; After}$$
  %\item[Note:] If in the each node there is only one class  represented, then $D_{\kappa}$ is zero.

\end{itemize}
\end{frame}

\begin{frame} % Cover slide
\frametitle{\textcolor{brique}{[ Example on a simple tree ]}}
\hfill \includegraphics[width = 0.3\textwidth]{rparttree-1.png}
\begin{itemize}
\item[]
  \only<2> {\includegraphics[width = 0.8\textwidth]{Step1-1.png} \\ }
  \only<2> {First the boundary decision line}
  \only<3> { \includegraphics[width = 0.8\textwidth]{Step1-2.png} \\ }
  \only<3> {The space below the line is classified as  rural}
  \only<4> { \includegraphics[width = 0.8\textwidth]{Step1-3.png} \\ }
  \only<4> {Some "Urban" are classified as  rural $\hookrightarrow$ Impurity}
  \only<5> { \includegraphics[width = 0.8\textwidth]{Step1-4.png} \\ }
  \only<5> {Some "Rural" are classified as  Urban $\hookrightarrow$ Impurity}
  \only<6> {\includegraphics[width = 0.8\textwidth]{Step2-1.png} \\ }
  \only<6> {Second boundary decision line}
  \only<7> {\includegraphics[width = 0.8\textwidth]{Step2-2.png} \\ }
  \only<7> {The space on the left of the line is classified as  rural}
  \only<8> {\includegraphics[width = 0.8\textwidth]{Step2-3.png} \\ }
  \only<8> {Some "Urban" are classified as  Rural }
  \only<9> {\includegraphics[width = 0.8\textwidth]{Step2-4.png} \\ }
  \only<9> {Some "Rural" are classified as  Urban}
  \only<10> {\includegraphics[width = 0.8\textwidth]{Step3-1.png} \\ }
  \only<10> {Third boundary decision line}
  \only<11> {\includegraphics[width = 0.8\textwidth]{Step3-2.png} \\ }
  \only<11> {The space on the right of the line is classified as  Urban}
  \only<12> {\includegraphics[width = 0.8\textwidth]{Step3-3.png} \\ }
  \only<12> {Some "Rural" are classified as  Urban }
  \only<13> {\includegraphics[width = 0.8\textwidth]{Step3-4.png} \\ }
  \only<13> {"Urban" seem well classified in the remaining space}
  \only<14> {\includegraphics[width = 0.8\textwidth]{Step4-1.png} \\ }
  \only<14> {Fourth boundary decision line}
  \only<15> {\includegraphics[width = 0.8\textwidth]{Step4-2.png} \\ }
  \only<15> {The space above the line is classified as  Urban}
  \only<16> {\includegraphics[width = 0.8\textwidth]{Step4-3.png} \\ }
  \only<16> {Finally, the remaining space is classified as rural}
  \only<17> {\includegraphics[width = 0.8\textwidth]{Step4-4.png} \\ }
  \only<17> {Few "Rural" are classified as  Urban }
  \only<18> {\includegraphics[width = 0.8\textwidth]{Step4-5.png} \\ }
  \only<18> {Many "Urban" are classified as Rural }

\end{itemize}
\end{frame}

\begin{frame}
\frametitle{\textcolor{brique}{[ How to build a tree? ]}}
The construction is based on recursive binary splits
\pause
\begin{itemize}[<+->]
    \item The goal is to increase the quality of the classification at the each stage
    \item[$\hookrightarrow$] decrease the \textit{impurity} at each node
    \item The outcome is "one" tree, not the perfect one
    \item  There are many parameters that can be adjusted
    \begin{itemize}[<+->]
        \item The maximum \textit{depth} of the tree
        \item The number of final leaves
        \item The impurity balance between classes
        \item The \textit{complexity parameter}
        \item ...
    \end{itemize}
\end{itemize}
\end{frame}

\section{Tuning a Tree}

\begin{frame}
\frametitle{\textcolor{brique}{[ Trees can be complex ]}}
Trees can decompose the space in very specific zones.\\
 $\hookrightarrow$ Example with  the full set of variables\\

\includegraphics[width = 0.9\textwidth]{Originaltree-1.png}
\end{frame}


\begin{frame}
\frametitle{\textcolor{brique}{[ Selecting the \textbf{Depth} of a tree ]}}
Using CV, we can select the \textit{maximum depth} parameter
\pause
\begin{itemize}[<+->]
    \item[] \includegraphics[width = 0.9\textwidth]{depthtune-1.png}
\end{itemize}
\end{frame}


\begin{frame}
\frametitle{\textcolor{brique}{[ Selecting the \textbf{Depth} of a tree ]}}
\pause
\begin{itemize}
\item[] The resulting tree
  \only<2> {\includegraphics[width = 0.8\textwidth]{depthtree-1.png} \\ }
  \only<2> {Tree with optimal depth}
  \only<3> {\includegraphics[width = 0.8\textwidth]{depthVIF-1.png} \\ }
  \only<3> {Feature importance (also confusion matrix, kappa, etc..)}
\end{itemize}
\end{frame}



\begin{frame}
\frametitle{\textcolor{brique}{[ Selecting the \textbf{Complexity} of a tree ]}}
The complexity of a tree is a parameter $C_p$ governing the trade-off between tree size $|T|$ and its overall
accuracy $D(T)$:
$$
D_{C_p}(T) = D(T) + C_p \cdot |T|
$$
\begin{itemize}[<+->]
    \item  $D(T) =  \sum_{\kappa=1}^{K} D_{\kappa}$:  the total \emph{impurity} of the tree
    \item $|T|$ is the number of terminal nodes of the tree
    \item[$\hookrightarrow$] A model with $C_p$ = 0 will impose no constrains
    \item[$\hookrightarrow$] A value of $C_p = 1$ only \textbf{one} terminal (and initial) node.
\end{itemize}
\end{frame}


\begin{frame}
\frametitle{\textcolor{brique}{[ Selecting the \textbf{Complexity} of a tree ]}}
\pause
\begin{itemize}
\item[] The result:
   \only<2> {\includegraphics[width = 0.8\textwidth]{Cptune-1.png} \\ }
   \only<2> {Grid search}
   \only<3> {\includegraphics[width = 0.8\textwidth]{Cptree-1.png} \\ }
   \only<3> {Final tree with optimal complexity parameter}
  \only<4> {\includegraphics[width = 0.8\textwidth]{depthVIF-1.png} \\ }
  \only<4> {Feature importance (also confusion matrix, kappa, etc..)}
\end{itemize}
\end{frame}


\begin{frame}
\frametitle{\textcolor{brique}{[ How to \textbf{prune} a tree? ]}}
One can also \textit{prune} a tree
\pause
\begin{enumerate}[<+->]
    \item Final trees may be too large and too complex
    \item[$\hookrightarrow$] Risk of overfitting
    \item Pruning techniques use the same criteria on each leave
    \item[$\hookrightarrow$] Remove least important nodes
\end{enumerate}
\end{frame}

\begin{frame}
\frametitle{\textcolor{brique}{[ Pruned tree ]}}
After "\textit{pruning}":
\pause
\begin{itemize}[<+->]
    \item[] \includegraphics[width = 0.9\textwidth]{prunedtree-1.png}
    \item[$\hookrightarrow$] Easier to interpret \& no loss of accuracy
\end{itemize}
\end{frame}

                  


\begin{frame} % Cover slide
\frametitle{\textcolor{brique}{[Quiz time]}}
\pause
\begin{itemize}[<+->]
  \item[]
\end{itemize}
\end{frame}

\section{Wrap-up}

\begin{frame} % Cover slide
\frametitle{\textcolor{brique}{[Takeaways]}}
\begin{itemize}[<+->]
\item Trees are simple and easy to interpret
\item Each node is a split based on a threshold
\item Splits are determined to maximize some measure of accuracy in each strata of the predictors' space
\item Regression trees apply the same logic, with different criteria and values
\item One can select the depth of a tree or its complexity and prune it
\item Trees are very specific, not robust and prone to overfitting
\item[$\hookrightarrow$] There are powerful methods using many trees...

\end{itemize}
\end{frame}








\end{document}
%%%%%%%%%%%%%%% Last Slide %%%%%%%%%%%%%%%%

\begin{frame}[allowframebreaks]%in case more than 1 slide needed
\frametitle{References}
    {\footnotesize
    %\bibliographystyle{authordate1}
    \bibliographystyle{apalike}
    \bibliography{../../../Visualisation/Visu}
    }
\end{frame}
\end{document}

%\bibliographystyle{authordate1}
%\bibliography{c:/Chris/Visualisation/Visu}
%\end{frame}
